% !Mode:: "TeX:UTF-8"%確保文檔utf-8編碼
%新加入的命令如下:addchtoc addsectoc reduline showendnotes hlabel
%新加入的环境如下:common-format  fig linefig xverbatim

\documentclass[11pt,oneside]{book}
\newlength{\textpt}
\setlength{\textpt}{11pt}


\usepackage{myconfig}
\usepackage{mytitle}


\begin{document}
\frontmatter

\titlea{The-Qt4-Book}
\author{Jasmin B., Mark S.}
\authorinfo{作者全名:Jasmin Blanchette, Mark Summerfield。本文代码来自\href{http://www.qtrac.eu/marksummerfield.html}{这个网站}。原中文翻译:闫锋欣,曾泉人,张志强;原中文审校:周莉娜,赵延兵。}
\editor{德山书生}
\email{a358003542@gmail.com}
\editorinfo{编者:德山书生,湖南常德人氏。我负责整理排版工作,关于版权我只有一句话,希望我的工作能够帮助更多的人的学习和研究。本项目Github网站在\href{https://github.com/a358003542/The-Qt4-Book}{这里}。有意见请反馈。}
\version{0.01}
\titleLA

\addchtoc{前言}
\chapter*{前言}
\begin{common-format}
为什么会是Qt?为什么像我这样的程序员会选择Qt?这个问题的答案显而易见:Qt单一源程序的兼容性、丰富的特性、C++方面的性能、源代码的可用性、它的文档、高质量的技术支持,以及在奇趣科技公司那些精美的营销材料中所涉及的其他优势等。这些答案看起来确实都不错,但是遗漏了最为重要的一点:Qt的成功缘于程序员们对它的喜欢。

那么,是什么让程序员喜欢某种技术而放弃另外一种呢?就我而言,我认为软件工程师们喜欢某种技术,是因为他们觉得这种技术是合适的,但是这也会让他们讨厌所有那些他们觉得不合适的其他技术。除此之外,我们还能解释下面的这些情况吗?例如,一些最出众的程序员需要在帮助之下才能编写出一个录像机程序,或者又比如,似乎大多数工程师在操作本公司的电话系统时总会遇到麻烦。我虽然善于记住随机数字和指令的序列,但是如果将其比作用于控制我的应答系统所需要的条件来说,则可能一条也不具备。在奇趣科技公司,我们的电话系统要求在拨打其他人的分机号码前,一定要按住"*"键2秒后才允许开始拨号。如果忘记了这样做而是直接拨打分机号码,那么就不得不再重新拨一遍全部的号码。为什么是"*"键而不是"\#{}"键、"1"键或者"5"键?或者为什么不是20个电话键盘中的其他任何一个呢?又为什么是2秒,而不是1秒、3秒或者1.5秒呢?问题到底出在哪里?我发现电话很气人,所以我尽可能不去使用它。没有人喜欢总是去做一些不得不做的随机事情,特别是当这些随机事情显然只出现在同样随机的情况下的时候,真希望自己从来都没有听到过它。

编程很像我们正在使用的电话系统,并且要比它还糟糕。而这正是Qt所要解决的问题。Qt与众不同。一方面,Qt很有意义;另一方面,Qt颇具趣味性。Qt可以让您把精力集中在您的任务上。当Qt的首席体系结构设计师面对一个问题的时候,他们不是寻求一个好的、快速的或者最简便的解决方案,而是在寻求一个恰当的解决方案,然后将其记录在案。应当承认,他们犯下了一些错误,并且还要承认的是,他们的一些设计决策没有通过时间的检验,但是他们确实做出了很多正确的设计,并且那些错误的设计应当而且也是能够进行改正的。看一看最初设计用于构建Windows 95和UNIX Motif之间的桥梁系统,到后来演变为跨越Windows Vista、Mac OS X和
GNU/Linux以及那些诸如移动电话等小型设备在内的统一的现代桌面系统,这些事实就足以证明这一点。

早在Qt大受欢迎并且被广泛使用很久以前,正是Qt的开发人员为寻求恰当的解决方案所做出的贡献才使Qt变得与众不同。其贡献之大,至今仍然影响着每一个对Qt进行开发和维护的人。对我们而言,研发Qt是一种使命和殊荣。能够使您的职业生涯和开源生活变得更为轻松和更加有趣,这让我们倍感自豪。

人们乐于使用Qt的诸多原因之一是它的在线帮助文档,但是该帮助文档的主要目的是集中介绍个别的类,而很少讲述应当如何构建现实世界中那些复杂的应用程序。这本好书填补了这一缺憾,它展示了Qt所提供的东西,如何使用“Qt的方式”进行Qt编程,以及如何充分地利用Qt。本书将指导C++、Java或者C\#{}程序员进行Qt编程,并且提供了丰富详实的资料来使他们成长为老练的Qt程序员。这本书包含了很多很好的例子、建议和说明——并且,该书也是我们对那些新加入公司的程序员们进行培训的入门教材。

如今,已有大量的商业或者免费的Qt应用程序可以购买或者下载,其中的一些专门用于特殊的高端市场,其他一些则面向大众市场。看到如此多的应用程序都是基于Qt构建而成的,这使我们充满了自豪感,并且还激励我们要让Qt变得更好。相信在这本书的帮助下,将会前所未有地出现更多的、质量更高的Qt应用程序。

{\hfill Matthias Ettrich}

{\hfill 德国,柏林}

{\hfill 2007年11月}

\section{序言}
Qt使用“一次编写,随处编译”的方式为开发跨平台的图形用户界面应用程序提供了一个完整的C++应用程序开发框架。Qt允许程序开发人员使用应用程序的单一源码树来构建可以运行在不同平台下的应用程序的不同版本;这些平台包括从Windows 98到Vista、MacOS X、Linux、Solaris、HP-UX以及其他很多基于X11的Unix。许多Qt库和工具也都是Qt/Embedded Linux的组成部分。Qt/Embedded Linux是一个可以在嵌入式Linux上提供窗口系统的产品。

本书的目标就是教您如何使用Qt4来编写图形用户界面程序。本书从"Hello Qt"开始,然后很快地转移到更高级的话题中,如自定义窗口部件的创建和拖放功能的提供等。通过本书的\href{http://www.informit.com/store/c-plus-plus-gui-programming-with-qt4-9780132354165}{互联网站点},您可以下载到一些作为本书文字补充材料的示例程序。附录A说明了如何下载和安装这些软件,其中包括一个用于Windows的C++免费编译器。

本书分为四部分。第一部分涵盖了在使用Qt编写图形用户界面应用程序时所必需的全部基本概念和练习。仅掌握这一部分中所蕴含的知识就足以写出实用的图形用户界面应用程序。第二部分进一步深入介绍了Qt的一些重要主题,第三部分则提供了更为专业和高级的材料。您可以按任意顺序阅读第二部分和第三部分中的章节,但这是建立在您对第一部分中的内容非常熟悉的基础之上的。第四部分包括数个附录,附录B说明了如何构建Qt应用程序,附录C则介绍了Qt Jambi,它是Java版的Qt。

本书的第一版建立在Qt 3版本的基础上,尽管已通过全书修订来反映那些很好的Qt4编程技术,但本书还是根据Qt4的模型,视图结构、新的插件框架、使用Qt/Embedded Linux进行嵌入式编程等内容而引入了一些新的章节和一个新的附录。作为第二版,本书充分利用了Qt 4.2和Qt 4.3中引入的新特性对其进行了彻底更新,并包含“自定义外观”和“应用程序脚本”两个新的章以及两个新的附录。原有的“图形”一章已经拆分为“二维”和“三维”两章,在它们中间,涵盖了新的图形视图类和QPainter的OpenGL后端实现。此外,在数据库、XML和嵌入式编程等几章中,还添加了许多新内容。

与本书的前两版一样,这一版的重点放在如何进行Qt编程的说明和许多真实例子的提供上,而不是对丰富的Qt在线文档的简单拼凑和总结。因为本书纯粹讲授的是Qt 4编程中的原理和实践知识,因而读者能够轻松学会将要出现在Qt 4.4、Qt 4.5以及Qt 4.x等后续版本中的15个Qt新模块。如果您正在使用的Qt版本恰好是这些后续版本中的一个,那么当然要阅读一下参考文档中的"What's New in Qt 4.x"一章,以便可以对那些可用的新特性有一个总体把握。

在写作本书的时候,是假定您已经具备了C++、Java或者C\#{}的基本知识。本书中的例子代码使用的是C++中的一个子集,从而避免了很多在Qt编程中极少使用的C++特性。在某些不可避免而必须使用C++高级结构的地方,会在使用时对其做出必要的解释。如果您对Java或者C\#{}已经非常熟悉但是对C++还知之不多甚至一无所知,那么建议您先阅读附录D。附录D提供了对C++较为充分的介绍,从而能够让您具有使用本书所必备的C++知识。对于C++中的面向对象编程更为全面的介绍,建议您阅读由P. J. Deitel和H. M. Deitel编著的"C++ How to Program"(Prentice Hall , 2007),以及由Stanley B. Lippman , Josée Lajoie和Barhara E. Moo编著的"C++ Primer"(Addison-Wesley , 2005)这两本书。

\section{Qt简史}
Qt框架首度为公众可用是在1995年5月。它最初由Haavard Nord(奇趣科技公司的CEO)和Eirik Chambe-Eng(公司总裁)开发而成。Haavard和Eirik在位于挪威特隆赫姆的挪威科技学院相识,在那里,他们都获得了计算机科学的硕士学位。

Haavard对C++图形用户界面开发的兴趣始于1988年,当时一家瑞典公司委托他开发一套C++图形用户界面框架。几年后,在1990年的夏天,Haavard和Eirik因为一个超声波图像方面的C++数据库应用程序而在一起工作。这个系统需要一个能够在UNIX、Macintosh和Windows上都能运行的图形用户界面。在那个夏天中的某天,Haavard和Eirik一起出去散步,享受阳光,当他们坐在公园的一条长椅上时,Haavard说:“我们需要一个面向对象的显示系统。”由此引发的讨论,为他们即将创建的面向对象的、跨平台的图形用户界面框架奠定了智力基础。

1991年,Haavard和Eirik开始一起合作设计、编写最终成为Qt的那些类。在随后的一年中,Eirik提出了“信号和槽”的设想——一个简单并且有效的强大的图形用户界面编程规范,而现在,它已经可以被多个工具包实现。Haavard实践了这一想法,并且据此创建了一个手写代码的实现系统。到1993年,Haavard和Eirik已经开发出了Qt的第一套图形内核程序,并且能够利用它实现他们自己的一些窗口部件。同年末,为了创建“世界上最好的C++图形用户界面框架”,Haavard提议一起进军商业领域。

1994年成为两位年轻程序员不幸的一年,他们没有客户,没有资金,只有一个未完成的产品,但是他们希望能够闯进一个稳定的市场。幸运的是,他们的妻子都有工作并且愿意为他们的丈夫提供支持。在这两年里,Haavard和Eirik认为,他们需要继续开发产品并且从中赚得收益。

之所以选择字母“Q”作为类的前缀,是因为该字母在Haavard的Emacs字体中看起来非常漂亮。随后添加的字母"t"代表“工具包”(toolkit),这是从"Xt"——一个X工具包的命名方式中获得的灵感。公司于1994年3月4日成立,最初的名字是"Quasar Technologies",随后更名为"Troll Tech",而公司今天的名字则是"Trolltech"。

1995年4月,通过Haavard就读过的大学的一位教授的联系,挪威的Metis公司与他们签订了一份基于Qt进行软件开发的合同。大约在同一时间,公司雇佣了Arnt Gulbrandsen,在公司工作的6年时间里,他设计并实现了一套独具特色的文档系统,并且对Qt的代码也做出了不少贡献。

1995年5月20日,Qt 0.90被上传到sunsite.unc.edu。6天后,在\\ comp.os.linux.announce上发布。这是Qt的第一个公开发行版本。Qt既可以用于Windows上的程序开发,又可以用于UNIX上的程序开发,而且在这两种平台上,都提供了相同的应用程序编程接口。从第一天起,Qt就提供了两个版本的软件许可协议:一个是进行商业开发所需的商业许可协议版,另一个则是适用于开源开发的自由软件许可协议版。Metis的合同确保了公司的发展,然而,在随后长达10个月的时间内,再没有任何人购买Qt的商业许可协议。

1996年3月,欧洲航天局(European Space Agency)购买了10份Qt的商业许可协议,它成了第二位Qt客户。凭着坚定的信念,Eirik和Haavard又雇佣了另外一名开发人员。Qt 0.97在同年5月底正式发布,随后在1996年9月24日,Qt1.0正式面世。到了这一年的年底,Qt的版本已经发展到了1.1,共有来自8个不同国家的客户购买了18份Qt的商业许可协议。也就是在这一年,在Matthias Ettrich的带领下,创立了KDE项目。

Qt 1.2于1997年4月发布。Matthias Ettrich利用Qt建立KDE的决定,使Qt成为Linux环境下开发C++图形用户界面的事实标准。Qt 1.3于1997年9月发布。

Matthias在1998年加入公司,并且在当年9月,发布了Qt 1系列的最后一个版本——V 1.40。1999年6月,Qt 2.0发布,该版本拥有一个新的开源许可协议——Q公共许可协议(QPL,Q Public License),它与开源的定义一致。1999年8月,Qt赢得了LinuxWorld的最佳库/工具奖。大约在这个时候,Trolltech Pty Ltd(澳大利亚)成立了。

2000年,公司发布了Qt/Embedded Linux,它用于Linux嵌入式设备。Qt/Em-bedded Linux提供了自己的窗口系统,并且可以作为X11的轻量级替代产品。现在,Qt/X11和Qt/Embedded Linux除了提供商业许可协议之外,还提供了广为使用的GNU通用公共许可协议(GPL: General Public License)。2000年底,成立了Trolltech Inc.(美国),并发布了Qtopia的第一版,它是一个用于移动电话和掌上电脑(PDA)的环境平台。Qt/Embedded Linux在2001年和2002年两次获得了LinuxWorld的"Best Embedded Linux Solution"奖,Qtopia Phone也在2004年获得了同样的荣誉。

2001年,Qt 3.0发布。现在,Qt已经可用于Windows、Mac OS X、UNIX和Linux(桌面和嵌入式)平台。Qt 3提供了42个新类和超过500 000行的代码。Qt 3是自Qt 2以来前进历程中最为重要的一步,它主要在诸多方面进行了众多改良,包括本地化和统一字符编码标准的支持、全新的文本查看和编辑窗口部件,以及一个类似于Perl正则表达式的类等。2002年,Qt 3赢得了Software Development Times的"Jolt Productivity Award"\footnote{Jolt大奖素有“软件业界的奥斯卡”之美誉,共设通用类图书、技术类图书、语言和开发环境、框架库和组件、开发者网站等十余个分类,每个分类设有一个“震撼奖”(Jolt Award)和三个“生产力奖”(Productivity Award)。一项技术产品只有在获得了Jolt奖之后才能真正成为行业的主流,一本技术书籍只有在获得了Jolt奖之后才能真正奠定其作为经典的地位。虽然Jolt奖项并不起决定作用,但它代表了某种技术趋势与潮流——译者注。}。

2005年夏,Qt 4.0发布,它大约有500个类和9000多个函数,Qt 4比以往的任何一个版本都要全面和丰富,并且它已经裂变成多个函数库,从而使开发人员可以根据自己的需要只连接所需要的Qt部分。相对于以前的所有Qt版本,Qt 4的进步是巨大的,它不仅彻底地对高效易用的模板容器、高级的模型/视图功能、快速而灵活的二维绘图框架和强大的统一字符编码标准的文本查看和编辑类进行了大量改进,就更不必说对那些贯穿整个Qt类中的成千上万个小的改良了。现如今,Qt 4具有如此广泛的特性,以至于Qt已经超越了作为图形用户界面工具包的界限,逐渐成长为一个成熟的应用程序开发框架。Qt 4也是第一个能够在其所有可支持的平台上既可用于商业开发
又可用于开源开发的Qt版本。  

同样在2005年,公司在北京开设了一家办事处,以便为中国及其销售区域内的用户提供服务和培训,并且为Qt/Embedded Linux和Qtopia提供技术支持。

通过获取一些非官方的语言绑定件(language bmdings),非C++程序员也已早就开始使用Qt,特别是用于Python程序员的PyQt语言绑定件。2007年,公司发布了用于C\#{}程序员的非官方语言绑定件Qyoto。同一年,Qt Jambi投放市场,它是一个官方支持的Java版Qt应用程序编程接口。附录C提供了对Qt Jambi的介绍。

自奇趣科技公司诞生以来,Qt的声望经久不衰,而且至今依旧持续高涨。取得这样的成绩不但说明了Qt的质量,而且也说明了人们都喜欢使用它。在过去的10年中,Qt已经从一个只被少数专业人士所熟悉的“秘密”产品,发展了到如今遍及全世界拥有数以千计的客户和数以万计的开源开发人员的产品。


\section{编者的话}
书名修改不是想标新立异,实在是github和本地文档编译方便,不支持空格。


%这里空一行。

\end{common-format}


\addchtoc{目录}
\setcounter{tocdepth}{2}
\tableofcontents

\begin{common-format}
\mainmatter


\part{Qt基础}
\chapter{Qt入门}
这一章介绍了如何把基本的C++知识与Qt所提供的功能组合起来创建一些简单的图形用户界面(Graphical User Interface,GUI)应用程序。在这一章中,还引入了Qt中的两个重要概念:一个是“信号和槽”,另外一个是“布局”。第2章还将对它们做进一步的阐述,而第3章将着手创建一个具有真正意义的应用程序。  

如果你已经熟知Java或C\#{},但对C++的编程经验还有些欠缺的话,那么在开始阅读本书之前,可能需要先阅读附录D,它对C++做了简要介绍。

\section{Hello Qt}
我们先从一个非常简单的Qt程序开始。首先一行一行地研究这个程序,然后将会看到如何编译并运行它。

\begin{tcbinput}[]{c++}{qt4-book/chap01/hello/hello.cpp}
\end{tcbinput}

第1行和第2行包含了类QApplication和QLabel的定义。对于每个Qt类,都有一个与该类同名(且大写)的头文件,在这个头文件中包括了对该类的定义。

第5行创建了一个QApplication对象,用来管理整个应用程序所用到的资源。这个QApplication构造函数需要两个参数,分别是argc和argv,因为Qt支持它自己的一些命令行参数。

第6行创建了一个显示“Hello Qt!”的QLabel窗口部件(widget)。在Qt和UNIX的术语(tenninology)中,窗口部件就是用户界面中的一个可视化元素。该词起源于“window gadget”(窗口配件)这两个词,它相当于Windows系统术语中的“控件”(control)和“容器”(container)。按钮、菜单、滚动条和框架都是窗口部件。窗口部件也可以包含其他窗口部件,例如,应用程序的窗口通常就是一个包含了一个QMenuBar、一些QToolBar、一个QStatusBar以及一些其他窗口部件的窗口部件。绝大多数应用程序都会使用一个QMainWindow或者一个QDialog来作为它的窗口,但Qt是如此灵活,以至于任意窗口部件都可以用作窗口。在本例中,就是用窗口部件QLabel作为应用程序的窗口的。   

第7行使QLabel标签(label)可见。在创建窗口部件的时候,标签通常都是隐藏的,这就允许我们可以先对其进行设置然后再显示它们,从而避免了窗口部件的闪烁现象。

第8行将应用程序的控制权传递给Qt。此时,程序会进入事件循环状态,这是一种等待模式,程序会等候用户的动作,例如鼠标单击和按键等操作。用户的动作会让可以产生响应的程序生成一些事件(event,也称为“消息”),这里的响应通常就是执行一个或者多个函数。例如,当用户单击窗口部件时,就会产生一个“鼠标按下”事件和一个“鼠标松开”事件。在这方面,图形用户界面应用程序和常规的批处理程序完全不同,后者通常可以在没有人为干预的情况下自行处理输入、生成结果和终止。

为简单起见,我们没有过多关注在main()函数末尾处对QLabel对象的delete操作调用。在如此短小的程序内,这样一点内存泄漏(memory  leak)问题无关大局,因为在程序结束时,这部分内存是可以由操作系统重新回收的。
\begin{fig}{helloqt}
\caption{Linux上的Hello程序}
\label{fig:Linux上的Hello程序}
\end{fig}








%这里空一行

\end{common-format}
\end{document}




这些操作会由Qt的安装程序自动完成。)还需要将该程序的源代码保存到hello.cpp文件,并把它
放进一个名为hello的目录中。你可以自行把代码录入到hello.cpp文件,也可以从与本书配套的
那些例子中复制该文件,它放在examples/chapOl/heUo/hello.cpp文件中。(所有例子都可以从本书
的网站中获取,网址是http://www .infonnit.com/title/01323541 60。)
    在命令提示符下,进入Fiello目录,输入如下命令,生成一个与平台无关的项目文件hello.pro:
    qmake -project
煞后,输入如下命令,从这个项目文件生成一个与平台相关的makefile文件:
    qmake hello.pro
    键入make命令就可以构建该程序。(在附录B中,会给出qmake工具更为详细的说明。)要运
行该程序,在Windows下可以输入hello,在岍职下可以输入./hello,在Mac OS x下可以输入open
 hello.app。要结束该程序,可直接单击窗口标题栏上的关闭按钮。
    如果使用的是Windows系统,并且已经安装了Qt的开源版和MinGW编译器,那么将会看到一
个指向MS-DOS提示符窗口的快捷键,其中已经正确地创建了使用Qt时所需的全部环境变量。,如
果启动了这个窗口,那么就可以在里面像上面所讲述的那样使用qmake命令和make命令编译Qt
应用程序。而由此产生的可执行文件将会保存在应用程序所在目录的debug或release文件夹中
(例如,C:\examplesV chapOl\hello\release\hello. exe).
  +女口果使用的是Microsoft VisLial c++和商业版的Qt,Uli懦要用nmake命令代替make命令。,除了这
一方法外,还可以遁过heⅡo.pro文件创建一个Visual Studio的工程文件,此时需要输入命令:,  .
    qmake -tp vc hello.pro
然后就可以在Visual Studio中编译这个程序了。如果使用的是Mac OS x系统中的Xcode,那各可以
使用如下命令来生成一个Xcode工程文件:
    qmake -spec macx-xcode hello.pro
    在开始进入下一个例子之前,我们一起来做一件有意思的事情:将代码行
    QLabel *label=new QLabel(”Hello  Qt rIt);
  替换为
    QLabel *label=new QLabel(“<h2×i>Hello</i>u    .t
    “<font  colo r=red>Qt!</font></h2>n);
    然后重新编译该程序。运行程序时,看起来应当是图1.2的样子。正如该例子所显示的那样,通
    过使用一些简单的HrML样式格式,就可以轻松地把Qt应用程序的用户接口变得更为丰富多彩。
}.
;\、    弘一爿,。构…(嘲秒、e+一叠
     4                                                  C++ GUI Qt 4编程(第 -版)
       -- --
-_---:----.
图1.2具有简单HTML样式的标签
1.2建立连接
    第二个确子要说明的是如何响应用户的动作。这个应用程序由一个按钮构成,用户可以单击
这个按钮退出程序。除了应用程序的主窗’口部件使用的是QPushButton而不是QLabel亭篓:兰竺
应用程序的源代码和Hello程序的源代码非常相似。同时,我们还会将用户的一个动作(单击按
钮)与一段代码连接起来。
    这个应用程序的源代码位于本书的例子文件中,文件名是exam-
ples/chapOI/quit/quit. cpp。程序的运行效果如图1.3所示。以下是该
文件所包含的内容:
    1  #include <QApplicat,ion>    图1.3 Quit应用程序
    2 #include <QPushButton>
    3  int main(int argc, char *argv【】)
    4  {
    5    QApplication app(argc,  argv);    ’
    6    0PushButton车button=new QPushButton(’‘Quit”);
    ,QObj ect::connec‘(bu‘‘on,—§{}:糍;j÷6“80‘’’r
    8    &app,SLO    );
    9    button->s胁w().  .    。
    10    returri app.exec(),
    11  )
    Qt的窗口部件通过发射信号(signal)来表明一个用户动作已经发生了或者是一个状态已经改
变了①。例如,当用户单击QPu。hButton时;该按钮就会发射一个chcked()信号。信号可以与函数
(在这里称为槽,slot)相连接,以便在发射焦号肘,f可以得赳自动执行。在这个例子中,我们把这
个按钮的chcked()信号与QApplication对象的quit()槽连接起来。宏SIGNAL(')和SLOT()是Qt语法
中的一部分。  .
    现在来构建这个应用程序。假设已经创建了一个包含quit.cpp文件的quit目录。在quit目录
中,首先运行qmake命令生成它的工程文件,然后再次运行该命令来生成一个makefile文件,这两
项操作的命令如下:
    qmake -project
    qmake quit.pro
    现在,就可以编译并运行这个应用程序了。如果单击Quit按钮,或者按下了空格键(这样也会
按下Quit按钮),那么将会退出应用程序。
1,3  窗口部件的布局
    这一节将创建一个简单的例子程序,以说明如何用布局(layout)来管理窗日中窗口部件的几何
形状,还要说明如何利用信号和槽来同步窗口部件。这个应用程序的运行效果如图1-4所示,它可
以用来询问用户的年龄,而用户可以通过操纵微调框( spin box)或者滑块(shder)来完成年龄的
输入。
①Qt的信号和LiMX的信号并不相关,本书中所讨论的信号仅指Q【信号。

